% HW Template for CS 6150, taken from https://www.cs.cmu.edu/~ckingsf/class/02-714/hw-template.tex
%
% You don't need to use LaTeX or this template, but you must turn your homework in as
% a typeset PDF somehow.
%
% How to use:
%    1. Update your information in section "A" below
%    2. Write your answers in section "B" below. Precede answers for all 
%       parts of a question with the command "\question{n}{desc}" where n is
%       the question number and "desc" is a short, one-line description of 
%       the problem. There is no need to restate the problem.
%    3. If a question has multiple parts, precede the answer to part x with the
%       command "\part{x}".
%    4. If a problem asks you to design an algorithm, use the commands
%       \algorithm, \correctness, \runtime to precede your discussion of the 
%       description of the algorithm, its correctness, and its running time, respectively.
%    5. You can include graphics by using the command \includegraphics{FILENAME}
%
\documentclass[11pt]{article}
\usepackage{amsmath,amssymb,amsthm}
\usepackage{graphicx}
\usepackage[margin=1in]{geometry}
\usepackage{fancyhdr}
\usepackage{framed}
\usepackage{algorithm}
\usepackage{algpseudocode}
\usepackage{pifont}
\setlength{\parindent}{0pt}
\setlength{\parskip}{5pt plus 1pt}
\setlength{\headheight}{13.6pt}
\newcommand\question[2]{\vspace{.25in}\textbf{#1: #2}\vspace{.5em}\vspace{.10in}}
\renewcommand\part[1]{\vspace{.10in}\textbf{(#1)}}
\newcommand\algorith{\vspace{.10in}\textbf{Algorithm: }}
\newcommand\correctness{\vspace{.10in}\textbf{Correctness: }}
\newcommand\runtime{\vspace{.10in}\textbf{Running time: }}
\pagestyle{fancyplain}
\lhead{\textbf{\NAME\ (\UID)}}
\chead{\textbf{HW\HWNUM \  Self Analysis}}
\rhead{CS 6300, \today}
\begin{document}\raggedright
%Section A==============Change the values below to match your information==================
\newcommand\NAME{Jake Pitkin}  % your name
\newcommand\UID{u0891770}     % your utah UID
\newcommand\HWNUM{0}              % the homework number
%Section B==============Put your answers to the questions below here=======================

\question{5.1}{What was the hardest part of the assignment for you?}

I didn't find this assignment challenging as I have experience using Python.

\question{5.2}{What was the easiest part of the assignment for you?}

The assignment was a simple warm-up that was quick to complete.

\question{5.3}{What problem(s) helped further your understanding of the course material?}

No course material was addressed in the assignment.

\question{5.4}{Did you feel any problems were tedious and not helpful to your understanding of the material?}

I didn't find any of the material tedious or not helpful.

\question{5.5}{What other feedback do you have about this homework?}

No other feedback for this warm-up assignment.



\end{document}