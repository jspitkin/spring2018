\documentclass[fleqn]{hermans-hw}

%\usepackage{haldefs}
\usepackage{amsmath}
\usepackage{float}
\usepackage{notes}
\usepackage{url}
\usepackage{graphicx}
\title{HW1: Search}
\duedate{}
\class{CS6300: Artificial Intelligence, Spring 2018}
\institute{University of Utah}
\author{Jake Pitkin}
% IF YOU'RE USING THIS .TEX FILE AS A TEMPLATE, PLEASE REPLACE
% The author WITH YOUR NAME AND UID.
% Replace the due date with anyone you worked with i.e. "Worked with: John McCarthy, Watson, & Hal-9000"

\begin{document}

\maketitle
\section{Graph Search}

For the following problems, S will denote Start and G will denote Goal. When choosing an arbitrary order of state expansions, alphabetical ordering will be used. Once visited, a state will not be expanded again.

\begin{enumerate}
\item Greedy search
\begin{table}[H]
\centering
{\renewcommand{\arraystretch}{1.2}%
\begin{tabular}{| c | c | c |}
\hline
\textbf{Step} & \textbf{Priority Queue} & \textbf{Expand}\\
\hline
1 & (S, 0) & S\\ \hline
2 & (S-A, 3), (S-B, 2) & B\\ \hline
3 & (S-A, 3), (S-B-D, 1), (S-B-C, 2), (S-B-G, 4) & D\\ \hline
4 & (S-A, 3), (S-B-C, 2), (S-B-G, 4), (S-B-D-G, 5) & C\\ \hline
5 & (S-A, 3), (S-B-G, 4), (S-B-D-G, 5), (S-B-C-G, 1) & G\\ \hline
\end{tabular}}
\caption{Greedy search}
\end{table}

\textbf{Greedy search final path: $S \rightarrow B \rightarrow C \rightarrow G$}

\item Depth first search
 
Note: a LIFO queue is used and sibling nodes are enqueued in alphabetical order.

\begin{table}[H]
\centering
{\renewcommand{\arraystretch}{1.2}%
\begin{tabular}{| c | c | c |}
\hline
\textbf{Step} & \textbf{LIFO Queue} & \textbf{Expand}\\
\hline
1 & S & S\\ \hline
2 & B, A & B\\ \hline
3 & G, D, C, A & G\\ \hline
\end{tabular}}
\caption{Depth first search}
\end{table}

\textbf{Depth first search final path: $S \rightarrow B \rightarrow G$}

\item Breadth first search

Note: a FIFO queue is used and sibling nodes are enqueued in alphabetical order. If a node is already in the queue, it is not re-added.

\begin{table}[H]
\centering
{\renewcommand{\arraystretch}{1.2}%
\begin{tabular}{| c | c | c |}
\hline
\textbf{Step} & \textbf{FIFO Queue} & \textbf{Expand}\\
\hline
1 & S & S\\ \hline
2 & A, B & A\\ \hline
3 & B, G & B\\ \hline
4 & G, C, D & G\\ \hline
\end{tabular}}
\caption{Breadth first search}
\end{table}

\textbf{Breadth first search final path: $S \rightarrow A \rightarrow G$}

\item Uniform cost search

\begin{table}[H]
\centering
{\renewcommand{\arraystretch}{1.2}%
\begin{tabular}{| c | c | c |}
\hline
\textbf{Step} & \textbf{Priority Queue} & \textbf{Expand}\\
\hline
1 & (S, 0) & S\\ \hline
2 & (S-A, 3), (S-B, 2) & B\\ \hline
3 & (S-A, 3), (S-B-C, 4), (S-B-D, 3), (S-B-G, 6) & A\\ \hline
4 & (S-B-C, 4), (S-B-D, 3), (S-B-G, 6), (S-A-G, 6) & D\\ \hline
5 & (S-B-C, 4), (S-B-G, 6), (S-A-G, 6), (S-B-D-G, 8) & C\\ \hline
6 & (S-B-G, 6), (S-A-G, 6), (S-B-D-G, 8), (S-B-C-G, 5) & G\\ \hline
\end{tabular}}
\caption{Uniform cost search}
\end{table}

\textbf{Uniform cost search final path: $S \rightarrow B \rightarrow C \rightarrow G$}

\item Greedy search with the heuristic values listed at each state

\begin{table}[H]
\centering
{\renewcommand{\arraystretch}{1.2}%
\begin{tabular}{| c | c | c |}
\hline
\textbf{Step} & \textbf{Priority Queue} & \textbf{Expand}\\
\hline
1 & (S, 0) & S\\ \hline
2 & (S-A, 2), (S-B, 3) & A\\ \hline
3 & (S-B, 3), (S-A-G, 0) & G\\ \hline
\end{tabular}}
\caption{Greedy search with heuristic values}
\end{table}

\textbf{Greedy search with heuristic final path: $S \rightarrow A \rightarrow G$}

\item A* search with the heuristic values listed at each state

\begin{table}[H]
\centering
{\renewcommand{\arraystretch}{1.2}%
\begin{tabular}{| c | c | c |}
\hline
\textbf{Step} & \textbf{Priority Queue} & \textbf{Expand}\\
\hline
1 & (S, 0) & S\\ \hline
2 & (S-A, 5), (S-B, 5) & A\\ \hline
3 & (S-B, 5), (S-A-G, 6) & B\\ \hline
4 & (S-A-G, 6), (S-B-C, 5), (S-B-D, 4), (S-B-G, 6) & D\\ \hline
5 & (S-A-G, 6), (S-B-C, 5), (S-B-G, 6), (S-B-D-G, 8) & C\\ \hline
6 & (S-A-G, 6), (S-B-G, 6), (S-B-D-G, 8), (S-B-C-G, 5) & G\\ \hline
\end{tabular}}
\caption{A* search}
\end{table}

\textbf{A* search final path: $S \rightarrow B \rightarrow C \rightarrow G$}

\end{enumerate}

\newpage
\section{Downhill Skiing}

\begin{enumerate}
\item If the mountain is $N$ units tall (eg., it is $N=6$ units tall in
the figure), what is the size of the state space?  Justify your
answer.  (You may ignore ``unreachable'' states.)  What are the
start/goal states?

Let's consider the state space $S$ of the example where $N = 6$. A state $s_i \in S$ is defined by a position and a current velocity as such: $s_i = (position, velocity)$. First we will define the start state $s_0$ and goal state $g$.

$$s_0 = (1, 0), \ g = (6, 0)$$

For small $N$, we can simply enumerate all the possible velocities at each position on the hill to determine the reachable states. Her action adjusts her velocity \textit{before} she moves squares, which is how a velocity of 1 at square 1 is possible. Also, Alice makes a final action at the goal state.

\begin{table}[H]
\centering
{\renewcommand{\arraystretch}{1.2}%
\begin{tabular}{| c | c | c | c |}
\hline
(1, 0) & (1, 1) & - & - \\ \hline
(2, 0) & (2, 1) & (2, 2) & - \\ \hline
(3, 0) & (3, 1) & (3, 2) & - \\ \hline
(4, 0) & (4, 1) & (4, 2) & (4, 3) \\ \hline
(5, 0) & (5, 1) & (5, 2) & (5, 3) \\ \hline
(6, 0) & (6, 1) & (6, 2) & (6, 3) \\ \hline
\end{tabular}}
\caption{State space for Alice's ski run.}
\end{table}

To justify this table, we will consider it one column at a time. 

\textit{Column 1:} Alice can be present in each square with a velocity of 0 by alternating \texttt{accelerate} and \texttt{decelerate} actions all the way down the hill.

\textit{Column 2:} Alice can \texttt{accelerate} at square 1 and \texttt{coast} the rest of the way down the hill.

\textit{Column 3:} From square $n - 1$ with a velocity of 0 before performing an action, \texttt{accelerate} twice. The state will be arrived at after the second \texttt{accelerate} but before moving.

\textit{Column 4:} From square $n - 2$ with a velocity of 1 before performing an action, \texttt{accelerate} twice. The state will be arrived at after the second \texttt{accelerate} but before moving.

There is no way to have a velocity greater than $3$ when $N = 6$. The best chance at the largest velocity is to \texttt{accelerate} for every action (as the higher on the hill you are the more space you have to accelerate before hitting the parking lot). Consider we \texttt{accelerate} 3 times. This will take us to space $(4, 3)$ after performing the action but before moving. Once we move, we will be overshooting the goal into the parking lot.

Thus, $\mathbf{|S| = 20}$.

\item Give an example of a state that is not reachable.  Suppose that
Alice cannot coast (she must either accelerate or decelerate): does this
yield \emph{more} unreachable states? If so, give an example of one and
justify your answer either way.

An example of a state that is not reachable is $(1, 2)$ (being at square 1 with a velocity of 2). Her velocity at square 1 is either $0$ (she \texttt{coasts} as her first move) or $1$ (she \texttt{accelerates} as her first move). There is no way for her to move back up the hill either.

When we remove the action of \texttt{coast}, the number of unreachable states does not change. In question one, I used \texttt{coast} to justify being at each square with a velocity of 1. Another way to achieve this without coasting is to alternate between \texttt{accelerate} and \texttt{decelerate}. This will place Alice in all squares with a velocity of 1.

\item Is Alice's current elevation (i.e., distance from the chair lift)
an admissible heuristic?  Why or why not?

A heuristic $h(n)$ is admissible if $h(n) \leq h*(n)$ where $h*(n)$ is the true cost to the nearest goal.

Alice's current elevation is an admissible heuristic, we can prove this by comparing the heuristic with the best true cost for each square on the hill.

\begin{table}[H]
\centering
{\renewcommand{\arraystretch}{1.2}%
\begin{tabular}{| c | c | c |}
\hline
\textbf{h(n)} & \textbf{h*(n)} & \textbf{Possible Best Path}\\
\hline
- & - & -\\ \hline
- & - & -\\ \hline
- & - & -\\ \hline
- & - & -\\ \hline
- & - & -\\ \hline
- & - & -\\ \hline
\end{tabular}}
\caption{Comparison of h(n) and h*(n) for $N \in [0, 5]$}
\end{table}


\item State and justify a non-trivial, admissible heuristic for this
problem which is \emph{not} current elevation.
\end{enumerate}

\end{document}
